\section{Relevante interesser}

\begin{itemize}
    \item \textbf{Sammenhænge:} \\
    En hel generel interesse er at vide hvordan ting hænger sammen. Det var roden til min interesse i fysik og springet til software.
    
    \item \textbf{Fysik:} \\
    Gennem gymnasiet blev jeg glad for fysikken og specielt logikken der indgik i den verden. Jeg besluttede efter gymnasiet at jeg ikke havde fået nok, og opgraderede derefter fra fysik B til fysik A. Dette ledte til DTU Fysik og Nanoteknologi, som jeg også har været meget glad for. I løbet af Fysik studiet blev jeg yderligere draget mod programmeringen og dens fleksibilitet, hvorimod fysikken syntes at skabe strammere og strammere rammer for hvad der var muligt. \\
    
    Min fysik-baggrund har specielt givet mig erfaring mht problem løsning og algebra hvilket altid er godt at have med.
    
    \item \textbf{Programmering:} \\
    Lige så snart jeg blev introduceret til programmering, har jeg fundet det mere og mere interessant. Selv styrker og svagheder blandt forskellige programmeringssprog giver ny variation af muligheder. Som tidligere nævnt er det betryggende at programmeringens relevans bliver mere og mere gældende.
    
    \item \textbf{Machine Learning, Neurale Netværk, Deep Learning, AI:} \\
    Et særligt punkt inden for programmering og eventuelt simulering, er kunstig intelligens. Måske er det bioteknologien fra gymnasiet, eller generel fascination af undervidstheden. Umiddelbart bliver kandidaten styret mod kunstig intelligens med stor spænding.
    
    \item \textbf{Internet:} \\
    Internettet er en af de "ting" som næsten alle bruger og næsten ingen ved hvordan virker. Det går ikke. Jeg \textit{må} vide hvordan internettet virker.
    
\end{itemize}